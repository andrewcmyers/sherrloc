In this paper, we introduce a general approach to diagnosing
program errors detected by type systems or other program analyses.
This approach works on any analysis that can be described as a
constraint system, in which a detected error corresponds to one or
more unsatisfiable constraints.  Both satisfiable and unsatisfiable
paths through the constraint system are analyzed, to identify the
program expressions most likely to be the cause of the unsatisfiable
constraints. The likelihood of different error explanations is
evaluated under the assumption that the programmer's code is mostly
correct, so the simplest error explanations are chosen, according to
maximum a posteriori principles. For analyses that depend on
programmer-stated assumptions about the environment, the error
diagnosis also identifies likely missing assumptions.  The new error
diagnosis approach has been implemented for two very different program
analyses: type inference in OCaml and information flow checking in
Jif. The effectiveness of the approach is evaluated using a corpus of
previously collected programs containing errors. The results show that
the technique identifies the location of program errors significantly
more accurately than in the original compilers, and also identifies
missing assumptions effectively.
