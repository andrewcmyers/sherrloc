%\usepackage[margin=1in]{geometry}
 
\usepackage{times,graphicx,color}
\usepackage{amsmath, amssymb, stmaryrd}
\usepackage{ttquot,utf8}
\usepackage[comments]{declarations}

\renewcommand{\floatpagefraction}{0.75}
\renewcommand{\dblfloatpagefraction}{0.75}

\title{Information Flow Graph}



\def\sharedaffiliation{%
\end{tabular}
\begin{tabular}{c}}
%


\numberofauthors{2}
\author{
\alignauthor Danfeng Zhang\\
\email{zhangdf@cs.cornell.edu}
%\affaddr{Department of Computer Science} \\
%\affaddr{Cornell University} \\
%\affaddr{Ithaca, NY, 14853}\\
%
\alignauthor Andrew C. Myers\\
\email{andru@cs.cornell.edu}
%\affaddr{Department of Computer Science} \\
%\affaddr{Cornell University} \\
%\affaddr{Ithaca, NY, 14853}\\
%
\sharedaffiliation
\affaddr{Department of Computer Science} \\
\affaddr{Cornell University} \\
\affaddr{Ithaca, NY 14853}
}

\begin{document}

%\maketitle

\begin{abstract}
flow graph
\end{abstract}

\section{Introduction}

flow graph
%\paragraph{Contributions}

This paper makes the following contributions:

\begin{enumerate}
\item
general representation

\item
application to error diagnosis

\item
evaluation
\end{enumerate}

\section{Model}
\label{sec:model}

Model: Figure~\ref{figure:lang:syntax}

\begin{figure}
\begin{align*}
C &::=\; \Gamma \proves E \leq E \ |\ C \land C\\
\Gamma &::= E \leq E \\
E &::= x\ |\ c(E_1,\dots,E_{a(c)})\ |\ c^{-i}(E)\ |\ E_1 \join E_2 \
|\ E_1 \meet E_2
\end{align*}
\caption{Syntax of constraints}
\label{figure:lang:syntax}
\end{figure}


set constraints or lattice? constructors?

We may not need constants, which is constructors with no parameters.

We assume elements of constraint $E$. Moreover, the constraints may
contain both join and meet operations (elements may form a lattice).
When join and meet are used, we assume for all $e_1, e_2 \in E, e_1
\join e_2 \in E \land e_1 \meet e_2 \in E$ to make the constraints
well-formed. In another word, the elements form a lattice. We define
atomic constraint $c$ in the form $e_1 \leq e_2$, where $e_1 \in E
\land e_2 \in E$. A constraint follows the form of $C_1 \proves C_2$.
For simplicity, we write the assumptions as $\G$ when it is clear.
$\Gamma$ is the assumption when the right hand is assumed to be valid.
Finally, the goal $G$ is a set of constraints. 

A goal is satisfiable when there is a valuation s.t. all constraints
are _valid_. Therefore, formally, the goal of this paper is to answer
which element or constraint to blame when a goal is unsatisfiable. A
possible reason may both be the existence of an element or the absence
of one.
\DZ{a missing condition (missing where clause) is common in Jif}

For _valid_, we assume an oracle $o(e_1, e_2, \G)$ that is true iff
$\G \proves e_1 \leq e_2$, where $e_1, e_2$ contain no variables. 

For type inference:

A valuation is a mapping of the type variables to _ground types_,
where ground types form a Herbrand universe.

Type inference is reduced to constraint solving by defining a mapping
of _pre-judgements_to constraints:
\[\trans{\G \proves x:\tau} = x =\tau\]
\[\trans{\G \proves \lambda x.e:\tau} = \exists a_1 a_2.
(def x: a_1 in \trans{e:a_2} \land a_1 \rightarrow a_2 =\tau)\]
\[\trans{\G\proves e_1 e_2 : \tau} = \exists a.(\trans{\G \proves
e_1:a\rightarrow \tau}\land \trans{\G\proves e_2:a})\]

\DZ{We may eliminate the $\exists$ by creating fresh variables}

\begin{figure}[b]
\begin{align*}
C &::=\; \tau = \tau\ |\ C \land C \\
\tau &::= x\ |\ \tau\rightarrow \tau
\end{align*}
\caption{Syntax of constraints}
\label{figure:cons:syntax}
\end{figure}

\paragraph{Examples}

For information flow control, elements are security labels, which in
nature forms a lattice~\cite{denning-lattice}. Join and meet
operations in constraints are consistent with that in the lattice.
$\leq$ is consistent with the partial order in lattice. Some languages
such as Jif allows ``where'' clauses that adding partial orders into
the lattice. The ``where'' clauses corresponds to the environments.
For other models, the environment is always empty.

Type inference can also naturally map into constraint solving. This
view is not new\DZ{REF}. For instance, Wand~\cite{wand-typeinference}
recast Hindley-Milner type system into equality constraints. Aiken and
Wimmers~\cite{aiken-typeinclusion} extends the Hindley-Milner type
system with inclusion constraints, and models function types,
constructor types, liberal intersection and union types. Moreover,
with sophisticated systems, constraint solvers are used for dependent
types \DZ{REF}, \DZ{and so on}. In this setting, types are the
elements. Join and meet operations are the usual intersection and
union types if that is well defined.  $\leq$ is consistent with the
subtype relationship between two labels.  Since the definition is
abstract, both syntactic subtyping \DZ{REF} and semantic
subtyping~\cite{aiken-typeinclusion} fit in this model.

\paragraph{Polymorphic Types}

Jif has polymorphic types, for instance, $caller\_pc$. Polymorphic
types are also crucial for ML-like languages.

\section{Graph} 

In this section, we formalize the constraint-solving problem into a
graph problem: to find a path in graph where no partial order on the
source and sink exist. The insight is consistent with ``A flow model
FM is _secure_ if and only if execution of a sequence of operations
cannot give rise to a flow that violates the relation
$\rightarrow$''~\cite{denning-lattice}.

First, for all the constraints in the right-hand-side, we
first split the constraints into _atomic_ form according to the
straightforward rules \DZ{only when the lattice is distributive}:

\[
\G \proves l_1 \meet l_2 \meet \dots \meet l_m \leq r_1 \join r_2 \join \dots
\join r_n
\]

For constraints in this form, we generate nodes for each side, and
add one _conditional edge_.

Static edges?

Back edges?

\DZ{No need} Conservative edge. When the right part constrains more
than one variable, in general, satisfiability would be undecidable. As
a common practice\DZ{true?}, we conservatively add edges from the
union node to all variable components.

\section{Related Work}

Set-constraint program analysis~\cite{aiken-setconstraint}.

Using graph reachability as program analysis is also not
new. Program slicing, shape analysis, flow-insensitive points-to
analysis are amenable to graph-reachability program~\cite{reps-graph}.

Interestingly, the equality between context-free-language reachability
(CFG-reachability) and a class of set constraints was also notice
before~\cite{melski-cflgraph}. But only a very restrict set of
constraints, namely the constraints $sexp \leq V$ where $V$ is a
set variable, are considered.

Policy inference~\cite{chong:sp11, harris:ccs10}.

Automatic instrumentation~\cite{king:esop10}.

DIFC OS?

Using statisctical analysis for bug finding is not new. Dawson Engler.

\bibliographystyle{abbrv}
\bibliography{constraint,../bibtex/pm-master}

\end{document}
